%%%%%%%%%%%%%%%%%%%%%%%%%%%%%%%%%%%%%%%%%%%%%%%%%%%%%%%%%%%%%%%%%%%%%%%%%%%%%%%%
%% encoding UTF-8
%%%%%%%%%%%%%%%%%%%%%%%%%%%%%%%%%%%%%%%%%%%%%%%%%%%%%%%%%%%%%%%%%%%%%%%%%%%%%%%%

\documentclass[
	english,%
	nologo,%
	noflag,%
	booktabs,%
	helvetica,%
	totpages%
]{curriculum}%

%%%%%%%%%%%%%%%%%%%%%%%%%%%%%%%%%%%%%%%%%%%%%%%%%%%%%%%%%%%%%%%%%%%%%%%%%%%%%%%%
%% Basic info
%%%%%%%%%%%%%%%%%%%%%%%%%%%%%%%%%%%%%%%%%%%%%%%%%%%%%%%%%%%%%%%%%%%%%%%%%%%%%%%%

\hypersetup{%
	pdfauthor   = {Walter Dal'Maz Silva},
	pdftitle    = {Curriculum Vitae - Walter Dal'Maz Silva},
	pdfsubject  = {Curriculum Vitae},
	pdfkeywords = {Engineer, PhD, Materials Science, Scientific Computing}
}%

\ecvaddress{69 cours Richard Vitton, Lyon 69003, France}%

\ecvmobile{(+33) 6 51 34 98 28 (please, prioritize e-mail)}%

\ecvemail{walter.dalmazsilva.manager@gmail.com}%

\ecvname{DAL'MAZ SILVA, Walter}%

\ecvfootername{DAL'MAZ SILVA, Walter}%

\ecvnationality{Brazilian}%

\ecvdateofbirth{25th May 1989}%

\ecvhomepage{www.linkedin.com/in/walter-dal-maz-silva/}

\date{2023-08-21}

%%%%%%%%%%%%%%%%%%%%%%%%%%%%%%%%%%%%%%%%%%%%%%%%%%%%%%%%%%%%%%%%%%%%%%%%%%%%%%%%
%% Document
%%%%%%%%%%%%%%%%%%%%%%%%%%%%%%%%%%%%%%%%%%%%%%%%%%%%%%%%%%%%%%%%%%%%%%%%%%%%%%%%

\begin{document}%
\begin{europasscv}%

\ecvpersonalinfo%

%$$$$$$$$$$$$$$$$$$$$$$$$$$$$$$$$$$$$$$$$$$$$$$$$$$$$$$$$$$$$$$$$$$$$$$$$$$$$$$$

\ecvsection{Profile and Motivation}%

\ecvitem{}{%
Please let me introduce myself. I am Walter Dal'Maz Silva, 34, married, Brazilian living in France for about 10 years now. I accumulate more than a decade of experience in industry, comprising a first experience in Oil\&Gas, a PhD dedicated to aeronautic and automobile applications, then a career in steel production with a high focus in applied Data Science, followed my current position as a simulation engineer in advanced ceramics. That includes a broad variety of fields, from raw materials manufacture to mechanical construction, both in production settings and research, and also intermediate production technical support.
%
\par\vskip6pt%
%
%\input{letter/middle-computational.tex}
\input{letter/middle-materials.tex}
%
\par\vskip6pt%
%
Fluent in Portuguese, French, and English, I am also self-taught on several subjects, always open to more knowledge. Me and my wife are active trekkers and via ferrata climbers. On my side, I also practice rock climbing and mountain biking since an early age. I am an active member of Club Alpin Français, taking part in activities on some of the above and started to get training to become an activity supervisor at the club. I am also interested in learning languages, cosmopolitan environments, physical sciences, programming, numismatics, and philately.
%
\par\vskip6pt%
%
I come to you through this letter to kindly request the study of my job application and reinforce my interest in the applied position. I strongly believe my past experiences and degrees provide an excellent foundation for the job's responsibilities. Please let me know if any supplementary information is required. You will find in the following sections preceding my career timeline, a self-assessment of relative mastery (filled bar size) across different skill groups. I can find flexible time slots for a phone or online discussion -- or why not in person.
%
\par\vskip6pt%
%
Yours sincerely,
\par%
Walter Dal'Maz Silva
}

%$$$$$$$$$$$$$$$$$$$$$$$$$$$$$$$$$$$$$$$$$$$$$$$$$$$$$$$$$$$$$$$$$$$$$$$$$$$$$$$

\clearpage

\ecvsection{Industry skills}%

\ecvitem{}{%
	\barplot{%
		{Materials and process specification procedures}/0.95,
		%
		{International standards (ISO, ASTM, NACE, API, AWS, ASME, DNV)}/0.9,
		%
		{Project management, planning, and quality tools (FMEA, 5S)}/0.8
	}
}%

\ecvsection{Materials skills}%

\ecvitem{}{%
	\barplot{%
		{Thermal and thermochemical processing of materials}/0.95,
		%
		{Materials selection for mechanical design}/0.95,
		%
		{Materials chemical and microstructural analysis}/0.85,
		%
		{Mechanical characterization of metals and ceramics}/0.8,
		%
		{Thermal (DSC, DTA, TG) and x-ray diffraction analyses}/0.8
	}
}%

\ecvsection{Computational skills}%

\ecvitem{}{%
	\barplot{%
		%
		{Scientific programming in Python, Julia, and C++}/0.95,
		%
		{Development of low-order process numerical models}/0.95,
		%
		{Materials thermodynamics with Thermo-Calc and OpenCALPHAD}/0.9,
		%
		{Data analysis, modeling, and computer vision}/0.9,
		%
		{Process CFD simulation with Ansys Fluent and OpenFOAM}/0.8
	}
}%

%$$$$$$$$$$$$$$$$$$$$$$$$$$$$$$$$$$$$$$$$$$$$$$$$$$$$$$$$$$$$$$$$$$$$$$$$$$$$$$$

\ecvsection{Academic}%

\myAcademic%
{2013 -- 2017}%
{Materials Science and Engineering PhD}%
{Nancy, France}%
{Université de Lorraine (UL)}%
{Belmonte, Thierry}

\myAcademic%
{2007 -- 2011}%
{Materials Engineering}%
{Florianópolis, Brazil}%
{Universidade Federal de Santa Catarina (UFSC)}%
{Maliska, Ana Maria}%

%$$$$$$$$$$$$$$$$$$$$$$$$$$$$$$$$$$$$$$$$$$$$$$$$$$$$$$$$$$$$$$$$$$$$$$$$$$$$$$$

\ecvsection{Languages}%

\ecvmothertongue{Brazilian Portuguese}%

\ecvlanguageheader%
\ecvlanguage{English}{\ecvCTwo}{\ecvCTwo}{\ecvCOne}{\ecvCOne}{\ecvCOne}%
\ecvlanguage{French}{\ecvCTwo}{\ecvCTwo}{\ecvCOne}{\ecvCOne}{\ecvCOne}%
\ecvlanguage{German}{\ecvATwo}{\ecvATwo}{\ecvATwo}{\ecvATwo}{\ecvATwo}%
\ecvlanguagefooter[10pt]

%$$$$$$$$$$$$$$$$$$$$$$$$$$$$$$$$$$$$$$$$$$$$$$$$$$$$$$$$$$$$$$$$$$$$$$$$$$$$$$$

\clearpage%

\ecvsection{Career}%

%$$$$$$$$$$$$$$$$$$$$$$$$$$$$$$$$$$$$$$$$$$$$$$$$$$$$$$$$$$$$$$$$$$$$$$$$$$$$$$$

\ecvtitle{02/2022 -- Present}{Imerys, Vaulx-Milieu -- France}%
%
\ecvitem{}{Introduced and internalized simulation activities in the group, leads simulation activities. Propose approaches, software, infrastructure, and other solutions to perform materials processing simulations. Vulgarize results and act as interface to subcontractors. Main achievements:}%
%
\ecvitem{$\bullet$}{Implemented an in-house state of the art rotary kiln model for the simulation of thermal cycles and energy balance for several calcination processes relevant to the group.}%

%$$$$$$$$$$$$$$$$$$$$$$$$$$$$$$$$$$$$$$$$$$$$$$$$$$$$$$$$$$$$$$$$$$$$$$$$$$$$$$$

\ecvtitle{05/2017 -- 01/2022}{ArcelorMittal, Maizières-les-Metz -- France}%
\ecvitem{}{Industrial research with focus on gas phase processes for surface treatment steel. The role consisted of project management, model conception, and other creative tasks. In this position Acted as digital transformation leader in a team of more than 60 people, proposing and putting in place solutions based on Data Science and Machine Learning. Main achievements:}%
%
\ecvitem{$\bullet$}{Creation of an atmosphere simulation tool for optimizing its management in continuous annealing furnaces, enabling process setup for third-generation of galvanized steel.}%
%
\ecvitem{$\bullet$}{Handled several production crises in company sites, making use of massive data analysis for the identification of root causes and developed solution strategies.}%
%
\ecvitem{$\bullet$}{Made digital transformation of R\&D a reality, raising awareness of needs and perspectives at all levels through the introduction of concrete time-saving solutions.}%

%$$$$$$$$$$$$$$$$$$$$$$$$$$$$$$$$$$$$$$$$$$$$$$$$$$$$$$$$$$$$$$$$$$$$$$$$$$$$$$$

\ecvtitle{11/2013 -- 11/2016}{IRT M2P, Metz -- France}%
%
\ecvitem{}{Research of thermochemical treatment of steels for aerospace and automobile transmission applications. Studied both gas phase and metallurgical aspects, providing advances in the comprehension of the temper behavior of the diffusion layer in these alloys and the kinetics of decomposition of precursors. Modeling of gas phase allowed for establishing a simplified kinetic mechanism of acetylene decomposition. Main achievements:}%
%
\ecvitem{$\bullet$}{Introduced and validated a simplified kinetics mechanism for the simulation of acetylene pyrolysis through CFD, enabling detailed process study.}%
%
\ecvitem{$\bullet$}{Identified an ordered lattice leading to an undocumented type of hardening of steel after annealing, showing the potential of high temperature nitriding.}%

%$$$$$$$$$$$$$$$$$$$$$$$$$$$$$$$$$$$$$$$$$$$$$$$$$$$$$$$$$$$$$$$$$$$$$$$$$$$$$$$

\ecvtitle{01/2012 -- 10/2013}{Aker Solutions, Curitiba -- Brazil}%
%
\ecvitem{}{Selection, compatibility analysis, and specification of materials for sub-sea oil \& gas production equipment such as Xmas Trees, Manifolds, etc. Specification and technical alignment with clad pipe induction bending suppliers for applications in aforementioned equipment. Support to the specification of welded joints and nickel alloy cladding. Technical and quality audit of raw material (forgings and bar) suppliers. Product study during RFQ phases. Main technical standards used for product specification: ASTM, ISO, DNV, ASME, API, Petrobras.}%

%$$$$$$$$$$$$$$$$$$$$$$$$$$$$$$$$$$$$$$$$$$$$$$$$$$$$$$$$$$$$$$$$$$$$$$$$$$$$$$$

\ecvtitle{02/2012 -- 12/2012}{Ensitec, Curitiba -- Brazil}%
%
\ecvitem{}{Taugh introductory Materials Science and processes to technicians.}%

%$$$$$$$$$$$$$$$$$$$$$$$$$$$$$$$$$$$$$$$$$$$$$$$$$$$$$$$$$$$$$$$$$$$$$$$$$$$$$$$

\ecvblueitem{Internships}{Laboratorial/industrial internships performed to obtain the Materials Engineer degree:}%

\ecvitem[5pt]{09/2011 -- 12/2011}{Aker Solutions, Curitiba, Brazil: selection and specification of metallic materials for subsea oil and gas equipment production.}%

\ecvitem[5pt]{06/2010 -- 12/2010}{Institut Jean Lamour, Nancy, France: study of decomposition of an organic material submitted to a microwave plasma post-discharge.}%

\ecvitem[5pt]{09/2009 -- 05/2010}{Aker Solutions, Curitiba, Brazil: specification of procedures for structural and overlay welds for the manufacture of subsea equipment. Selection and specification of polymeric materials.}%

\ecvitem[5pt]{02/2009 -- 05/2009}{SteelInject Injeção de Aços Ltda, Caxias do Sul, Brazil: study of shape retention in parts produced from a new polymeric binding system for powder injection molding technology.}%

\ecvitem[5pt]{05/2008 -- 09/2008}{Materials Laboratory -- UFSC, Florianópolis, Brazil: treatment of materials by glow discharges -- nitriding of precipitation hardening stainless steels and sintering of ferrous alloys.}%

\ecvitem{Other activities}{Responsible for the practical lectures on metallography and optical microscopy during academic periods (when not in internship, as listed above) between February 2008 and August 2011.}

%$$$$$$$$$$$$$$$$$$$$$$$$$$$$$$$$$$$$$$$$$$$$$$$$$$$$$$$$$$$$$$$$$$$$$$$$$$$$$$$

\clearpage%

\ecvsection{Papers}%

\ecvitem[10pt]{}{F.~Cavilha Neto, T.~Bendo, B.~Borges Ramos, W.~Dal’Maz Silva, C.~Binder, A.~N.~Klein, ``Controlled addition of air in the gas mixture of plasma nitriding : an analysis of nitrided layer microstructure and microhardness of carbon steels,'' {\em Journal of the Brazilian Society of Mechanical Sciences and Engineering}, vol.~44:199, 2022.}%

\ecvitem[10pt]{}{W.~Dal’Maz Silva, J.~Dulcy, J.~Ghanbaja, A.~Redjaïmia, G.~Michel, S.~Thibault and T.~Belmonte, ``Carbonitriding of low alloy steels : Mechanical and metallurgical responses,'' {\em Materials Science and Engineering : A}, vol.~693, pp.~225–232, 2017.}%

\ecvitem[10pt]{}{W.~D. Silva, J.~Dulcy, G.~Michel, S.~Thibault, P.~Lamesle, and T.~Belmonte,``Carbonitruration des aciers faiblement alliés -- réponses à la trempe et au revenu,'' {\em Traitements et Matériaux}, no.~438, pp.~30--37, 2016.}%

\ecvitem[10pt]{}{W.~Dal'Maz~Silva, T.~Belmonte, D.~Duday, G.~Frache, C.~No{\"e}l, P.~Choquet, H.-N. Migeon, and A.~M. Maliska, ``Interaction mechanisms between \ch{Ar} -- \ch{O2} post-discharge and biphenyl,'' {\em Plasma Processes and Polymers}, vol.~9, no.~2, pp.~207--216, 2012.}%

\ecvitem[10pt]{}{K.~C. Kleinjohann, M.~B. Martins, W.~D. Silva, B.~B. Ramos, and A.~M. Maliska,``Nitreta{\c c}{\~a}o por plasma de liga \ch{Ni-Cr-Mo} - Inconel 625,'' in {\em XIX Congresso Brasileiro de Engenharia e Ci{\^e}ncia dos Materiais (CBECIMat), Campos do Jord{\~a}o}, vol.~CD-ROM, 2010.}%

\ecvitem[10pt]{}{E.~A. Bernardelli, C.~Brunetti, J.~K. Brasil, W.~D. Silva, and A.~M. Maliska, ``Efeito da temperatura de solubiliza{\c c}{\~a}o no tratamento de  envelhecimento do a{\c c}o inoxid{\'a}vel 15-5 PH, envelhecido em forno mufla ou em reator de plasma.,'' in {\em XVII Congresso Brasileiro de Engenharia e Ci{\^e}ncia dos Materiais (CBECIMat), Porto de Galinhas}, vol.~CD-ROM, 2008.}%

%$$$$$$$$$$$$$$$$$$$$$$$$$$$$$$$$$$$$$$$$$$$$$$$$$$$$$$$$$$$$$$$$$$$$$$$$$$$$$$$

\ecvsection{Conferences}%

\ecvitem[10pt]{}{12$^{th}$ International Conference on Zinc and Zinc Alloy Coated Steel. Virtual Conference, 21st to 23rd June 2021. Oral presentation. W. Dal'Maz Silva, H. Saint-Raymond, C. Dulcy. An integrated methodology for the root cause analysis of mechanical and metallurgical defects in hot-dip galvanized coatings.}

\ecvitem[10pt]{}{43$^e$ Congrès du Traitement Thermique et de l'Ingénierie des Surfaces. Nancy, France, 8th and 9th June 2016.  Oral presentation. W. Dal'Maz Silva, J. Dulcy, J. Ghanbaja, G. Michel, P. Lamesle, T. Belmonte. Carbonitruration des aciers faiblement alliés: rôles du carbone et de l'azote sur les réponses mécaniques et métallurgiques.}

\ecvitem[10pt]{}{7$^{th}$ International Conference on Inovations in Thin Film Processing and Characterization, Nancy, France, 16th to 20th November 2015. Oral presentation. W. Dal'Maz Silva, J. Dulcy, G. Michel, P. Lamesle, T. Belmonte. The roles of carbon and nitrogen on metallurgical response of low alloy steels to carbonitriding.}

\ecvitem[10pt]{}{42$^e$ Congrès du Traitement Thermique et de l'Ingénierie des Surfaces. Saint-Etienne, France, 2nd to 4th June 2015. Oral presentation. W. Dal'Maz Silva, J. Dulcy, G. Michel, P. Lamesle, T. Belmonte. Traitements thermochimiques des alliages 16NiCrMo13 et 23MnCrMo5: le rôle du carbone et de l'azote sur les réponses métallurgiques à la carbonitruration.}

\ecvitem[10pt]{}{22$^{nd}$ International Federation for Heat Treatment and Surface Engineering Congress, Mestre, Italy, 20th to 22nd May 2015. Oral presentation. W. Dal'Maz Silva, J. Dulcy, G. Michel, P. Lamesle, T. Belmonte. Thermochemical treatments of alloys 16NiCrMo13 and 23MnCrMo5: the roles of carbon and nitrogen on metallurgical response to carbonitriding.}%

%$$$$$$$$$$$$$$$$$$$$$$$$$$$$$$$$$$$$$$$$$$$$$$$$$$$$$$$$$$$$$$$$$$$$$$$$$$$$$$$

\end{europasscv}%
\end{document}%