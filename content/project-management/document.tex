%%%%%%%%%%%%%%%%%%%%%%%%%%%%%%%%%%%%%%%%%%%%%%%%%%%%%%%%%%%%%%%%%%%%%%%%%%%%%%%
%% PROJECT MANAGEMENT
%%%%%%%%%%%%%%%%%%%%%%%%%%%%%%%%%%%%%%%%%%%%%%%%%%%%%%%%%%%%%%%%%%%%%%%%%%%%%%%

\documentclass[aspectratio=169,pdf]{beamer}
\usepackage{tcolorbox}
\usepackage{tikz}
\usetikzlibrary{mindmap}

%%%%%%%%%%%%%%%%%%%%%%%%%%%%%%%%%%%%%%%%%%%%%%%%%%%%%%%%%%%%%%%%%%%%%%%%%%%%%%%
%% TITLE
%%%%%%%%%%%%%%%%%%%%%%%%%%%%%%%%%%%%%%%%%%%%%%%%%%%%%%%%%%%%%%%%%%%%%%%%%%%%%%%

\title{Project construction and management}
\subtitle[]{Building solid numerical solutions}
\author{Walter Dal'Maz Silva}

%%%%%%%%%%%%%%%%%%%%%%%%%%%%%%%%%%%%%%%%%%%%%%%%%%%%%%%%%%%%%%%%%%%%%%%%%%%%%%%
%% COMMANDS
%%%%%%%%%%%%%%%%%%%%%%%%%%%%%%%%%%%%%%%%%%%%%%%%%%%%%%%%%%%%%%%%%%%%%%%%%%%%%%%

% Automatically adds section name to title and use title as subtitle.
\newenvironment{Frame}[1]{\begin{frame}{\secname}{#1}}{\end{frame}}

% Creates section and automatically add section page.
\newcommand{\Section}[1]{\section{#1}\frame{\sectionpage}}

%%%%%%%%%%%%%%%%%%%%%%%%%%%%%%%%%%%%%%%%%%%%%%%%%%%%%%%%%%%%%%%%%%%%%%%%%%%%%%%
%% BODY
%%%%%%%%%%%%%%%%%%%%%%%%%%%%%%%%%%%%%%%%%%%%%%%%%%%%%%%%%%%%%%%%%%%%%%%%%%%%%%%

%- Functional analysis
%- Life profile
%- Building
%- Storing
%- Transporting
%- Installing
%- Usign
%- Maintaining
%- Dismantling
%- Recycling
%
%- Functions
%- Failure modes
%- Failure consequences
%- Failure detection
%- Actions
%- Probability
%
%- Budget and resources
%- Invention Patents


\begin{document}

\frame{\titlepage}
\frame{\tableofcontents}

%%+++++++++++++++++++++++++++++++++++++++++++++++++++++++++++++++++++++++++++++
%%
%%+++++++++++++++++++++++++++++++++++++++++++++++++++++++++++++++++++++++++++++

{% Governance
\Section{Governance}

\begin{Frame}{About numerical projects}
The present document will treat only matters concerning project creation and
life-cycle, and outcomes on the following domains:
\vspace{3mm}
\begin{itemize}
	\item Numerical simulation (continuum mechanics, process-level, etc)
	\item Machine learning applied to technical and scientific fields
\end{itemize}

\vfill%

\begin{tcolorbox}[title=Terminology usage]
In the context of \emph{machine learning} we will refrain from using \emph{AI}
or associated megatrend jargon as they are misleading and do not seem perennial.
\end{tcolorbox}
\end{Frame}

\begin{Frame}{Project actors}
A project is built through collaboration between the following actors:
\vspace{3mm}
\begin{itemize}
\item Requestors (researchers, process engineers, plant managers, ...)

\item Management team (N+1, N+2, ..., depending on breadth of scope)

\item Numerical team (simulation or machine learning engineer(s), interns ...)

\item Technical body (field specialist(s), researchers, academic partners, ...)

\item IT (cloud support, infrastructure, security, ...)

\item Others based on subject-specific needs: experimental support (laboratory technicians, plant technicians, ...), field measurements (internal or third-party measurement team), automation (controls expert, deployment technicians, ...), intellectual property, ...
\end{itemize}
\vfill%
\end{Frame}

\begin{Frame}{Supported actions}
There must be a clear and transparent set of rules to define whether the project is executed internally or by a subcontractor if a decision has been made to pursue the idea goals, with or without modifications
\vfill%

The minimal set of requirements to perform in-house should include:
\begin{itemize}
\item Availability of human and computational resources to respond within deadlines
\item Existing technical expertise on the subject (numerical team and technical body)
\item Unless very high stakes are associated with the idea, the solution must be reusable
\item Innovation comes first when competing with technical support for resources
\end{itemize}
\end{Frame}

\begin{Frame}{Classes of projects}
	\vspace{-1cm}
	\begin{center}
		\includegraphics[scale=0.6]{media/project-classes}
	\end{center}
\end{Frame}
}% Governance

%%+++++++++++++++++++++++++++++++++++++++++++++++++++++++++++++++++++++++++++++
%%
%%+++++++++++++++++++++++++++++++++++++++++++++++++++++++++++++++++++++++++++++

{% Project life-cycle
\Section{Project life-cycle}

\begin{Frame}{Phases of a project}
\vspace{-1cm}
\begin{center}
\includegraphics[scale=0.6]{media/project-phases}
\end{center}
\end{Frame}

\begin{Frame}{Idea}
A project is born from an idea, generally raised by the \emph{need} to solve a problem
\vfill%

It is up to the \emph{requestor} to elaborate and provide a proper description of the idea
\vfill%

The \emph{requestor} presents the idea for appreciation of the \emph{management team}
\vfill%

\begin{itemize}
\item What is the problem the idea is trying to solve?

\item Is it something new or a modification of the existing?

\item Did you check the solution is not available in the market?

\item Do you have any tracks on how you intend to solve it?

\item Who are the people who \emph{need} to get involved?

\item ...
\end{itemize}
\end{Frame}

\begin{Frame}{Evaluation}
The \emph{management team} evaluates the resources availability and consults with
the \emph{numerical team} the technical feasibility and capacity of executing the idea
\vfill%

Together with the \emph{technical body}, the team assesses on whether the project
must be executed, strictly following a prescribed methodology
\vfill%

The activities must include at least
\begin{itemize}
\item A life-cycle assessment of the system to be simulated / modeled

\item The interaction with other parts and potential effects of solution implementation

\item Identification of alternatives in case of failure to perform numerical activity
\end{itemize}
\end{Frame}

\begin{Frame}{Execution}
Execution is performed by the \emph{numerical team} within the scope defined duing
the evaluation stage with the other actors
\vfill%

During this phase, the \emph{numerical team} may request support from \emph{IT} or
others depending on the scope of the study
\vfill%

The \emph{requestor} and \emph{technical team} follow-up the progress of the
developments and make sure they comply with initial goals and assessment
\vfill%

Upon completion the \emph{numerical team} documents all findings and models for
restitution to the other actors
\vfill%
\end{Frame}

\begin{Frame}{Closure}
Project closure comprises the transfer of solution from \emph{numerical team} to
the \emph{requestor}, who will manage the roll-out and next steps

\end{Frame}

\begin{Frame}{Outcomes}
Outcomes are evaluated by the \emph{requestor} with support of other parts and
presented to appreciation of \emph{management team}
\vfill%

The assessment must include potential of new projects, intellectual property,
transversal deployment or transfer to other plants,...
\vfill%
\end{Frame}

\begin{Frame}{Summary}
\end{Frame}


}% Project life-cycle

%%+++++++++++++++++++++++++++++++++++++++++++++++++++++++++++++++++++++++++++++
%%
%%+++++++++++++++++++++++++++++++++++++++++++++++++++++++++++++++++++++++++++++

{% Building a project
\Section{Building a project}

\begin{Frame}{xxx}
	content...
\end{Frame}
}% Building a project

%%+++++++++++++++++++++++++++++++++++++++++++++++++++++++++++++++++++++++++++++
%%
%%+++++++++++++++++++++++++++++++++++++++++++++++++++++++++++++++++++++++++++++

{% Managing a project
\Section{Managing a project}

\begin{Frame}{xxx}
	content...
\end{Frame}
}% Managing a project

\end{document}

%%%%%%%%%%%%%%%%%%%%%%%%%%%%%%%%%%%%%%%%%%%%%%%%%%%%%%%%%%%%%%%%%%%%%%%%%%%%%%%
%% EOF
%%%%%%%%%%%%%%%%%%%%%%%%%%%%%%%%%%%%%%%%%%%%%%%%%%%%%%%%%%%%%%%%%%%%%%%%%%%%%%%