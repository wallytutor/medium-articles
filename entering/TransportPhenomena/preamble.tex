\chapter*{Preâmbulo}%
\addcontentsline{toc}{chapter}{Preâmbulo}%

Anos atrás, durante meu doutorado, me propus à elaborar uma forma mais refinada dos famosos \href{https://jose.theoj.org/papers/10.21105/jose.00021}{12 Passos para Navier-Stokes~\footnote{Em inglês CFD Python: the 12 steps to Navier-Stokes equations, por Lorena A. Barba and Gilbert F. Forsyth.}}. Aquela tentativa não era uma crítica a publicação em si, a qual contribuiu no seu tempo com meu próprio aprendizado, mas à estrutura dos programas apresentados. Ademais, me parece demasiadamente mecanística a repetição dos elementos primeiramente em uma dimensão e então em duas dimensões. Outro problema que evoquei foi a ausência de um texto externo aos programas, que foi encapulado completamente num contexto de programção letrada~\footnote{Um conceito aonde o programa e o conteúdo que o explica devem encontrar-se juntos.}. Ao concluir meu trabalho, durante um tempo o mantive em minha página pessoal que acabou por ser descontinuada. Esse texto não é uma reescritura da referida publicação, mas uma abordagem alternativa combinando ideias provindas daquela com algumas opiniões pessoais. Com essa publicação tenho por objetivo de perenisar os materiais produzidos, mas também tornar sua manutenção ao longo dos anos melhor estruturada. Espero que os materiais sejam úteis tanto no ensino como no aprendizado de computação científica.

Este livro-texto é focado num primeiro contato com métodos numéricos aplicados a engenharia, mais particularmente nas áreas mecânica e química, mas pode ser útil como material recapitulativo ou mesmo em outras áreas. O conteúdo é apresentado de uma forma tendendo ao informal, e o rigor matemático não é um objetivo. Alguns prerequisitos são necessários para um bom proveito dos conteúdos, o que inclui ao menos um curso de \emph{Cálculo I} e \emph{Álgebra linear}. Embora o título trate de \emph{Fenômenos de transporte}, um curso de \emph{Mecânica de Fluidos} ou \emph{Transferência de Calor} não é requerido para se acompanhar o texto. Os elementos de base serão apresentados nos primeiros capítulos e na sequência os termos constituentes da equação de Navier-Stokes serão progressivamente apresentados até que alcancemos um nível de maturidade para resolvê-las em casos facilmente implementáveis. Para concluir, vamos abordar uma introdução ao métodos dos volumes finitos. Na prática este é o método utilizado pela maioria dos programas em código aberto ou comerciais para solução de equações de transporte, as razões serão apresentadas no tempo devido. Recomenda-se ao estudante aplicado de retrabalhar todo o conteúdo computacional do livro empregando este método ao fim da leitura para um aprendizado verdadeiramente profissionalizante.

Finalmente necessito justificar a escolha do ambiente computacional adotado, a linguage de programção \href{https://julialang.org/}{Julia}. Em 2016 decidi realizar meus primeiros passos em Julia e acabei decepctionado. A compilação \emph{just-in-time (JIT)}~\footnote{As vezes chamada \emph{tradução dinâmica} em português.} ainda não estava madura, o que causava muita frustração quanto a performance da linguagem. Ademais, as interfaces não eram estáveis e tudo mudava muito rapidamente na linguagem. Decidi de continuar com Python para todas minhas aplicações de alto nível e todo ano desde então passei alguns dias testando Julia novamente dado o forte potencial que vi na linguagem. Foi somente agora em 2023 que terminei por me convencer e praticamente abandonar meu uso de Python para computação científica. Embora já há alguns anos em versão superior à 1.0, foi somente recentemente que as interfaces realmente se estabilizaram e as principais extensões da linguagem (pacotes) estão funcionais em todos os principais sistêmas operacionais (Windows, Mac OS e Linux). Esse último critério é importante para o ensino, visto que torna-se extremamente complicado ocupar-se simultanêamente da reprodutibilidade dos ambientes de cálculos dos estudantes. Além do material suplementar a este livro, vários recursos de aprendizado de Julia podem ser encontrados em \href{https://juliaacademy.com/}{Julia Academy}. A linguagem vem ganhando momento na comunidade científica graças ao grande suporte provido pelo MIT, aonde foi criada. Para saber mais, o estudante interessado pode verificar \href{https://computationalthinking.mit.edu/}{este curso}.



\endinput%
